\documentclass{article}
\usepackage{graphicx}
\usepackage[spanish,activeacute]{babel}
\usepackage[utf8]{inputenc}
\usepackage{tgbonum}
\usepackage[margin=0.5in]{geometry}
\begin{document}
\begin{titlepage}
	\centering
	\includegraphics[width=2.5cm]{unr.png}\par\vspace{1cm}
	{\scshape \fontsize{8}{4} \selectfont {FACULTAD DE CIENCIAS EXACTAS, INGENIERÍA Y AGRIMENSURA} \par}
	\vspace{1cm}
	{\scshape\Large SISTEMAS OPERATIVOS I\par}
	\vspace{1.5cm}
	{\huge \bfseries TRABAJO PRÁCTICO FINAL\par}
	\vspace{4cm}
	\large \fontsize{12}{5} \selectfont Andrés Imlauer \par
\end{titlepage}
\noindent
{\Large \textbf{Introducción}} \\
En este informe se explicará a grandes rasgos la arquitectura del servidor y del cliente de \\
Ta-Te-Ti, además se explicarán como compilar, conectar nodos e iniciar el servidor, para
finalizar las futuras posibles mejoras al servidor. \\
El servidor y el cliente consisten de los siguientes archivos: client.erl, dispatcher.erl, game\_functions.erl, pbalance.erl,
pcomando.erl, psocket.erl, pstat.erl, server.erl. \\ \\
{\Large \textbf{Arquitectura del servidor}} \\
La función que comienza a realizar la interacción con el cliente es \textbf{dispatcher:start}, que espera
nuevas conexiones, luego crea un nuevo proceso llamado loop, donde se 
creará un nuevo hilo \textbf{psocket} por cada consulta del cliente, a su vez \textbf{psocket} por cada 
pedido del cliente crea un nuevo proceso \textbf{pcomando}, éste para no sobrecargar ningún 
nodo utiliza una funcion \textbf{nodo\_libre}, que retorna el nodo con menos carga en el 
momento que el cliente hace la petición, luego \textbf{psocket} inicia \textbf{pcomando} en el nodo 
anteriormente seleccionado. Finalmente la función \textbf{pcomando} ejecuta el comando 
enviado por el cliente. \\
Cada nodo posee dos procesos, \textbf{pstat} se encarga de mandar intervalos regulares la 
información de carga del nodo al resto, \textbf{pbalance}, recibe esta información y calcula cual 
nodo debe recibir los siguientes comandos.  \\ \\
{\large \textbf{Implementación de las funciones encargadas de realizar una nueva partida}} \\ 
Cuando un usuario hace la peticion para crear una nueva partida usando el comando 
\textit{new} lo que se hace es agregar la partida a una lista de juegos disponibles en el servidor, 
luego llama a la funcion \textbf{administrador}, lo que hace esta funcion es llamar a una nueva 
funcion llamada \textbf{nueva\_partida}, cuando me encuentro en \textbf{nueva\_partida} espero que el 
oponente me mande el comando \textit{acc} (por lo tanto si no se acepta la jugada no se puede 
jugar aunque la partida este en la lista de juegos disponibles), luego de que el usuario 
acepte la jugada se creará una nueva función llamada \textbf{espero\_guardo}, la cual mantiene 
una lista local por cada partida creada en distintos nodos. \\ \\
{\large \textbf{Implementación del cliente }} \\
En lo que respecta al cliente, es bastante sencillo, su funcionamiento sólo consiste en 
identificar los comandos del usuario, a estos agregarle un número de identificación, 
enviarlos y esperar las respuestas del servidor con la función \textbf{receive\_data}.  \\ \\
{\large \textbf{Compilando e iniciando el servidor y el cliente} } \\
En este ejemplo, iniciaremos un servidor local con dos nodos: \\
Primero, para iniciar el intérprete de Erlang escribimos erl y compilaremos el servidor escribiendo:
c(servidor). \\
Luego abrimos diferentes terminales en la misma ubicación donde escribiremos: erl -sname ast y en la otra terminal:
erl -same linus, también se puede usar: erl -name ast@IP. \\
Para conectar a los dos nodos, escribimos: net\_adm:ping('linus@tu\_host'), nodes() para comprobar que los nodos esten
conectados. \\
Ahora si todo fue bien, escribimos:
servidor:dispatcher\_start(número de Puerto) en cada terminal utilizando diferentes
puertos, y el servidor ya debería comenzar a funcionar. \\
Ahora el cliente, lo compilamos con c(client) , y nos conectamos a cualquiera de los
dos nodos con \\ 
client:start("127.0.0.1",Nº de Puerto). \\
La documentación de como jugar está explicada en el enunciado del trabajo, con lo que
se refiere a los comandos disponbiles puedes consultarlo enviando el comando HELP
desde el cliente. \\ \\
{\large \textbf{TODO}}
\begin{itemize} 
  \item Implementar una función para que elija todo el tiempo partidas disponibles
automáticamente, cuando finalice se conecte a otra, de no existir ninguna partida,
creala. Sin necesidad de interactuar con el usuario.
	\item Agregar una funcion para dar turnos de forma aleatoria.· Desarrollar un cliente que posea entorno gráfico
	\item Implementar un comando para ver perfiles de cada usuario, manteniendo información
de por ejemplo la fecha de registro, país, partidas perdidas, ganadas, si abandona
muchas partidas, etc.
	\item Agregar función para que el cliente se actualice automáticamente.
  \item Agregar tiempo límite.
  \item Chequear que el usuario esté conectado fuera de cada comando.
  \item Escribir el cliente usando escript
\end{itemize}
\end{document}
